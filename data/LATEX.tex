% Tecnológico de Costa Rica 
% Sistemas Operativos Avanzados 
% 3er proyecto 
% Scheduling en Tiempo Real 
 
\documentclass{beamer} 
\usetheme[progressbar=frametitle]{metropolis} 
\setbeamertemplate{frame numbering}[fraction] 
\useoutertheme{metropolis} 
\useinnertheme{metropolis} 
\usefonttheme{metropolis} 
\usecolortheme{metropolis} 
\usepackage[utf8]{inputenc} 
\usepackage{lmodern} 
\usepackage[T1]{fontenc} 
\usepackage{tikz} 
\usepackage{natbib} 
\usepackage{hyperref} 
\usepackage{multirow} 
\usepackage{colortbl} 
\usepackage{helvet} 
\usepackage[export]{adjustbox} % loads also graphicx 
\usepackage{lipsum} 
%Definiciones 
\definecolor{color_columna_candidata}{rgb}{0, 0.424, 0.455} 
\definecolor{color_pivote}{rgb}{0.973, 0.80, 0.341} 
\definecolor{color_blanco}{rgb}{1,1,1} 
% Commands 
\newcommand\tab[1][1cm]{\hspace*{#1}}  
\newcommand\minitab[1][0.5cm]{\hspace*{#1}}  
% Tittle information 
\title{Scheduling en Tiempo Real} 
\subtitle{Sistemas Operativos Avanzados} 
\author[A. \& D. \& E.]{% 
\texorpdfstring{% 
\begin{columns} 
\column{.33\linewidth} 
\centering 
\\  Nicole Carvajal \\ 
\column{.33\linewidth} 
\centering 
\\  Rubén González \\ 
\column{.33\linewidth} 
\centering 
\\ Edisson López \\ 
\end{columns} 
\begin{columns} 
\column{.2\linewidth} 
\centering 
\\  Otto Mena \\ 
\column{.2\linewidth} 
\centering 
\\  Cristina Soto  \\ 
\end{columns} 
} 
{Author 1, Author 2, Author 3} 
} 
\date{} 
\institute{% 
\texorpdfstring{% 
\begin{columns} 
\column{.9\linewidth} 
\centering 
\\ 
Tecnológico de Costa Rica \\ 
Maestría de Ciencias de la Computación \\ 
Semestre 1, 2021 
\end{columns} 
} 
} 
%Inicio del documento 
\begin{document} 

% - - 1st Slide - - ; 
% - - Cover - - - - ; 
\begin{frame}[plain,t] 
\maketitle 
\end{frame} 


% - - - - - - - - - ;
% - - - - 2 - - - - ;
% Breve introducción del proyecto 
\begin{frame}{Scheduling en Tiempo Real}
Simulación del comportamiento de varios algoritmos de scheduling clásicos para Sistemas Operativos de Tiempo Real (RTOS). Con una interaz gráfica hecha con GTK y generación de una presentación Beamer como salida.
\end{frame}

\begin{frame}{Algoritmo: Rate Monotonic} 
\begin{itemize} 
   \item Propuesto por Liu y Layland (1973) 
   \item Scheduling dinámico de tiempo  
   \item Es óptimo. 
   \item Expropiativo, de mayor prioridad primero 
   \item La prioridad de una tarea es inversamente proporcional a su periodo 
\end{itemize} 
\end{frame} 
\begin{frame}{Algoritmo: Earliest Deadline First} 
\begin{itemize} 
   \item Propuesto por Liu y Layland (1973) 
   \item Scheduling dinámico de tiempo real 
   \item Es óptimo. 
   \item Expropiativo, de mayor prioridad primero 
   \item La prioridad de una tarea es inversamente proporcional al tiempo pendiente para que se dé su deadline 
\end{itemize} 
\end{frame} 
\begin{frame}{Algoritmo: Least Laxity First} 
\begin{itemize} 
   \item Propuesto por Leung (1989) 
   \item Scheduling dinámico de tiempo real 
   \item Es óptimo. 
   \item Expropiativo, de mayor prioridad primero 
   \item La prioridad de una tarea es inversamente proporcional a su laxity. El laxity de la tarea i, d es el deadline, c es el tiempo de computación y t es el momento en el tiempo, se calcula: \( L_{i} = d_{i} - t_{i} - c_{i}\)  
\end{itemize} 
\end{frame} 
\begin{frame}{Tests de Schedulability} 
\begin{small} 
\textbf{Test de Liu y Layland} 
\begin{itemize} 
\item Fórmula: \( \mu = \sum{ c_{i} / p_{i}}  \leq U(n) =  n(2^{1/n} - 1)  \) 
\item Resultado del test RM: \( \mu = 0,94\> U(n) = 0,83)\), \textbf{rechazada}. 
\item Usando RM, puede que para este conjunto de tareas ocurra un incumplimiento del deadline o puede que no.
\item Resultado del test EDF: \( \mu = 0,94\leq 1 \), \textbf{aprobada}. 
\item Este conjunto de tareas definitivamente sí correrán con el algoritmo EDF.
\end{itemize} 
\textbf{Test de Bini} 
\begin{itemize} 
\item Fórmula: \( \mu = \prod{ (c_{i} / p_{i}} + 1 ) \leq 2\) 
\item Resultado del test: \( \mu =2,17\> 2 \), \textbf{rechazada}. 
\item Puede que para este conjunto de tareas ocurra un incumplimiento del deadline o puede que no.
\end{itemize} 
\end{small} 
\end{frame} 
\begin{frame}{Ejecución RM} 
\begin{center} 
\begin{tabular}{cccccccccccccccccc} 
\hline 
\multicolumn{1}{|l|}{\cellcolor[HTML]{34FF34}} &\multicolumn{1}{|l|}{\cellcolor[HTML]{34FF34}} &\multicolumn{1}{|l|}{\cellcolor[HTML]{34FF34}} &\multicolumn{1}{|l|}{} &\multicolumn{1}{|l|}{} &\multicolumn{1}{|l|}{} &\multicolumn{1}{|l|}{\cellcolor[HTML]{34FF34}} &\multicolumn{1}{|l|}{\cellcolor[HTML]{34FF34}} &\multicolumn{1}{|l|}{\cellcolor[HTML]{34FF34}} &\multicolumn{1}{|l|}{\cellcolor[HTML]{000000}} &\multicolumn{1}{|l|}{} &\multicolumn{1}{|l|}{} &\multicolumn{1}{|l|}{} &\multicolumn{1}{|l|}{} &\multicolumn{1}{|l|}{} &\multicolumn{1}{|l|}{} &\multicolumn{1}{|l|}{} &\multicolumn{1}{|l|}{} \\ \hline 
\multicolumn{1}{|l|}{} &\multicolumn{1}{|l|}{} &\multicolumn{1}{|l|}{} &\multicolumn{1}{|l|}{\cellcolor[HTML]{FE0000}} &\multicolumn{1}{|l|}{\cellcolor[HTML]{FE0000}} &\multicolumn{1}{|l|}{\cellcolor[HTML]{FE0000}} &\multicolumn{1}{|l|}{} &\multicolumn{1}{|l|}{} &\multicolumn{1}{|l|}{} &\multicolumn{1}{|l|}{\cellcolor[HTML]{000000}} &\multicolumn{1}{|l|}{} &\multicolumn{1}{|l|}{} &\multicolumn{1}{|l|}{} &\multicolumn{1}{|l|}{} &\multicolumn{1}{|l|}{} &\multicolumn{1}{|l|}{} &\multicolumn{1}{|l|}{} &\multicolumn{1}{|l|}{} \\ \hline 
\end{tabular}% 
\end{center} 
\end{frame}
\begin{frame}{Ejecución EDF} 
\begin{center} 
\begin{tabular}{cccccccccccccccccc} 
\hline 
\multicolumn{1}{|l|}{\cellcolor[HTML]{34FF34}} &\multicolumn{1}{|l|}{\cellcolor[HTML]{34FF34}} &\multicolumn{1}{|l|}{\cellcolor[HTML]{34FF34}} &\multicolumn{1}{|l|}{} &\multicolumn{1}{|l|}{} &\multicolumn{1}{|l|}{} &\multicolumn{1}{|l|}{} &\multicolumn{1}{|l|}{\cellcolor[HTML]{34FF34}} &\multicolumn{1}{|l|}{\cellcolor[HTML]{34FF34}} &\multicolumn{1}{|l|}{\cellcolor[HTML]{34FF34}} &\multicolumn{1}{|l|}{} &\multicolumn{1}{|l|}{} &\multicolumn{1}{|l|}{} &\multicolumn{1}{|l|}{} &\multicolumn{1}{|l|}{\cellcolor[HTML]{34FF34}} &\multicolumn{1}{|l|}{\cellcolor[HTML]{34FF34}} &\multicolumn{1}{|l|}{\cellcolor[HTML]{34FF34}} &\multicolumn{1}{|l|}{} \\ \hline 
\multicolumn{1}{|l|}{} &\multicolumn{1}{|l|}{} &\multicolumn{1}{|l|}{} &\multicolumn{1}{|l|}{\cellcolor[HTML]{FE0000}} &\multicolumn{1}{|l|}{\cellcolor[HTML]{FE0000}} &\multicolumn{1}{|l|}{\cellcolor[HTML]{FE0000}} &\multicolumn{1}{|l|}{\cellcolor[HTML]{FE0000}} &\multicolumn{1}{|l|}{} &\multicolumn{1}{|l|}{} &\multicolumn{1}{|l|}{} &\multicolumn{1}{|l|}{\cellcolor[HTML]{FE0000}} &\multicolumn{1}{|l|}{\cellcolor[HTML]{FE0000}} &\multicolumn{1}{|l|}{\cellcolor[HTML]{FE0000}} &\multicolumn{1}{|l|}{\cellcolor[HTML]{FE0000}} &\multicolumn{1}{|l|}{} &\multicolumn{1}{|l|}{} &\multicolumn{1}{|l|}{} &\multicolumn{1}{|l|}{} \\ \hline 
\end{tabular}% 
\end{center} 
\end{frame}
\begin{frame}{Ejecución LLF} 
\begin{center} 
\begin{tabular}{cccccccccccccccccc} 
\hline 
\multicolumn{1}{|l|}{\cellcolor[HTML]{34FF34}} &\multicolumn{1}{|l|}{\cellcolor[HTML]{34FF34}} &\multicolumn{1}{|l|}{\cellcolor[HTML]{34FF34}} &\multicolumn{1}{|l|}{} &\multicolumn{1}{|l|}{} &\multicolumn{1}{|l|}{} &\multicolumn{1}{|l|}{} &\multicolumn{1}{|l|}{\cellcolor[HTML]{34FF34}} &\multicolumn{1}{|l|}{\cellcolor[HTML]{34FF34}} &\multicolumn{1}{|l|}{\cellcolor[HTML]{34FF34}} &\multicolumn{1}{|l|}{} &\multicolumn{1}{|l|}{} &\multicolumn{1}{|l|}{} &\multicolumn{1}{|l|}{} &\multicolumn{1}{|l|}{\cellcolor[HTML]{34FF34}} &\multicolumn{1}{|l|}{\cellcolor[HTML]{34FF34}} &\multicolumn{1}{|l|}{\cellcolor[HTML]{34FF34}} &\multicolumn{1}{|l|}{} \\ \hline 
\multicolumn{1}{|l|}{} &\multicolumn{1}{|l|}{} &\multicolumn{1}{|l|}{} &\multicolumn{1}{|l|}{\cellcolor[HTML]{FE0000}} &\multicolumn{1}{|l|}{\cellcolor[HTML]{FE0000}} &\multicolumn{1}{|l|}{\cellcolor[HTML]{FE0000}} &\multicolumn{1}{|l|}{\cellcolor[HTML]{FE0000}} &\multicolumn{1}{|l|}{} &\multicolumn{1}{|l|}{} &\multicolumn{1}{|l|}{} &\multicolumn{1}{|l|}{\cellcolor[HTML]{FE0000}} &\multicolumn{1}{|l|}{\cellcolor[HTML]{FE0000}} &\multicolumn{1}{|l|}{\cellcolor[HTML]{FE0000}} &\multicolumn{1}{|l|}{\cellcolor[HTML]{FE0000}} &\multicolumn{1}{|l|}{} &\multicolumn{1}{|l|}{} &\multicolumn{1}{|l|}{} &\multicolumn{1}{|l|}{} \\ \hline 
\end{tabular}% 
\end{center} 
\end{frame}
\end{document}
% } DOCUMENT 
% Última línea del documento